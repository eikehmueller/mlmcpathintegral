\documentclass[11pt]{article}
\usepackage{amssymb,amsmath}
\usepackage[cm]{fullpage}
\renewcommand{\vec}[1]{\boldsymbol{#1}}
\title{Quantum mechanical rotor}
\begin{document}
\maketitle
\section*{Problem}
We consider the discretised action in \cite{Ammon2016}
\begin{equation}
S[\phi] = \frac{I}{a}\sum_{i=0}^{d-1} \left(
1-\cos(\phi_{i+1}-\phi_i)
\right)\label{eqn:action_discrete}
\end{equation}
for the state variable $\phi_i\in[-\pi,\pi)$ for $i=0,\dots,d-1$ with periodic boundery conditions. $I=M_0R^2$ is the angular moment and $a$ the lattice spacing. Since
\begin{equation}
\frac{1}{a^2}\left(1-\cos(\phi_{i+1}-\phi_i)\right) =
\frac{1}{2}\left(\frac{\phi_{i+1}-\phi_i}{a}\right)^2 + a^2 \left(
\frac{\phi_{i+1}-\phi_i}{a}\right)^4 + \mathcal{O}(a^4),
\end{equation}
Eq. \eqref{eqn:action_discrete} is a discretisation of the continuum action
\begin{equation*}
S_{\text{cont}}(\phi) = \frac{I}{2}\int_{0}^{T} \left(\frac{d\phi}{dt}\right)^2\;dt\qquad\text{with $T=a\cdot d$}
\end{equation*}
which naturally constrains the angle to $[-\pi,\pi)$.
\section*{Fine-level sampling}
To generate the conditioned fine level sample at the odd sites, we need to sample from a distribution with the probability density
\begin{equation*}
p(\phi|\phi_+,\phi_-) = \mathcal{Z}^{-1}\exp\left[-\frac{I}{a}f(\phi;\phi_+,\phi_-)\right]
\end{equation*}
with
\begin{equation}
f(\phi;\phi_+,\phi_-) = 2-\cos(\phi_+-\phi)-\cos(\phi-\phi_-).
\label{eqn:f}
\end{equation}
where $\mathcal{Z}=\mathcal{Z}(\phi_+,\phi_-)$ is a normalisation constant.
We can now rewrite Eq. \eqref{eqn:f} as follows:
\begin{equation*}
\begin{aligned}
f(\phi;\phi_+,\phi_-) &= 2 - \cos(\phi_+)\cos(\phi)-\sin(\phi_+)\sin(\phi) - \cos(\phi_-)\cos(\phi)-\sin(\phi_-)\sin(\phi)\\
&= 2 - (\cos(\phi_+)+\cos(\phi_-))\cos(\phi) - (\sin(\phi_+)+\sin(\phi_-))\sin(\phi)\\
&= 2 - C \cos(\phi-\delta \phi)
\end{aligned}
\end{equation*}
for some constants $C=C(\phi_+,\phi_-)$ and $\delta\phi=\delta\phi(\phi_+,\phi_-)$. More specifically
\begin{xalignat*}{2}
\tan(\delta\phi) &= \frac{\sin(\phi_+)+\sin(\phi_-)}{\cos(\phi_+)+\cos(\phi_-)},&
C &= 2 \left|\cos\left(\frac{\phi_+-\phi_-}{2}\right)\right|.
\end{xalignat*}
Inserting this into Eq. \eqref{eqn:f} gives
\begin{equation*}
  f(\phi;\phi_+,\phi_-) = 2-C+2C\sin^2\left(\frac{\phi-\delta\phi}{2}\right).
\end{equation*}
Hence, the density we need to sample from can be written as
\begin{equation*}
p(\phi|\phi_+,\phi_-) = \tilde{\mathcal{Z}}^{-1}\exp\left[
-\frac{2IC}{a} \sin^2\left(\frac{\phi-\delta\phi}{2}\right)
\right].
\end{equation*}
Due to the obvious symmetries, it would be sufficient if we could sample from the density
\begin{equation}
p_\sigma(\psi) = \mathcal{Z}^{-1}_{\sigma} \exp\left[
-\sigma \sin^2\left(\frac{\psi}{2}\right)
\right]\qquad\text{for $\psi\in[-\pi,\pi]$ and $\sigma=\frac{2IC}{a}\ge 0$.}
\label{eqn:dist_psi}
\end{equation}
The normalisation constant $\mathcal{Z}_\sigma$ can be calculated by observing that
\begin{equation*}
  \mathcal{Z}_\sigma = \int_{-\pi}^{\pi} \exp\left[-\sigma\sin^2\left(\frac{\psi}{2}\right)\right] = 2\pi \exp\left[-\frac{\sigma}{2}\right]I_0\left(\frac{\sigma}{2}\right)
\end{equation*}
where $I_0$ is the modified Bessel function of the first kind. We might now sample approximately from the distribution in Eq. \eqref{eqn:dist_psi} by replacing $\sin^2(\psi/2)\mapsto \psi^2/4$ in the exponential, to obtain a Gaussian distribution with width $\sqrt{\frac{2}{\sigma}}$:
\begin{equation*}
  p_{\text{Gauss},\sigma} = \mathcal{Z}_{\text{Gauss},\sigma}^{-1} \exp\left[-\frac{\sigma}{4}\psi^2\right]\qquad\text{with $\mathcal{Z}_{\text{Gauss},\sigma}^{-1}=\sqrt{\frac{\sigma}{4\pi}}$}.
\end{equation*}
This is a good approximation if $\sigma$ is large (more specifically, $\sigma^{-1/2}\ll\pi$), but will not work so well for small value of $\sigma$, which can occur if $\phi_+-\phi_-\approx \pi$. Note that $\sigma\propto a^{-1}$, and hence for small lattice spacings $a$ Gaussian sampling is likely to become increasingly better (but we might still get some bad distributions from $\phi_+-\phi_-\approx\pi$).

Another possible solution is rejection sampling. For this, choose a number of points $\psi_i$, $i=0,\dots,M$ such that $\psi_0=0$, $\psi_M=\pi$ and construct the approximate, piecewise constant density
\begin{equation}
q_\sigma(\psi) = Z_{q,\sigma}^{-1} \exp\left[
-\sigma \sin^2\left(\frac{\psi_j}{2}\right)
\right]\qquad\text{for $\psi\in[\psi_{j},\psi_{j+1})$.}
\end{equation}
The normalisation constant can be calculated by observing that
\begin{equation*}
\sum_{j=0}^{M-1} (\psi_{j+1}-\psi_j) q_\sigma(\psi_j) = 1.
\end{equation*}
Since $\sin^2(\psi/2)$ is a montonously increasing function of $\psi$, we have that $p_\sigma(\psi)\le q_\sigma(\psi)$. Hence, rejection sampling would work like this:
\begin{enumerate}
\item Draw a sample $\hat{\psi}$ from $q_\sigma$. Since this is a piecewise constant distribution, this can be done very easily.
  \item Accept this sample with the probability $p_\sigma(\hat{\psi})/q_\sigma(\hat{\psi})$. Otherwise, continue with Step 1 (and repeat until a sample is accepted).
\end{enumerate}
The points $\psi_i$ should probably be chosen such that they are clustered closer together where the slope of $\sin^2(\psi/2)$ is large. This will ensure that the ratio $p_\sigma(\psi)/q_\sigma(\psi)$ is not too far from one, i.e. we accept as many points as possible.
\bibliographystyle{unsrt}
\bibliography{rotor}
\end{document}
